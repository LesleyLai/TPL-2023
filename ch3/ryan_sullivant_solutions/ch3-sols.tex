\documentclass{article}

\usepackage{amsmath, amsfonts, amssymb}
\usepackage{amsthm}
\usepackage{enumerate}
\usepackage[cm]{fullpage}
\usepackage{xcolor}
\usepackage{graphicx}
\usepackage{caption}
\usepackage{subcaption}
\usepackage{siunitx}
\usepackage{multirow}
\usepackage{booktabs}
\usepackage{titlesec}
\usepackage{mathpartir}



\newtheorem{thm}{Theorem}
\newtheorem{lem}{Lemma}
\newtheorem{claim}{Claim}

\newcommand{\set}[2]{\{#1\mid#2\}}
\newcommand{\rr}{\mathbb{R}}
\newcommand{\ee}{\mathbb{E}}
\newcommand{\pp}{\mathbb{P}}
\newcommand{\nn}{\mathbb{N}}

\newcommand{\comment}[1]{\textcolor{red}{\textbf{Comment:} #1}}
\newcommand{\ifthenelse}[3]{\mathtt{if}\,\,\mathtt{#1}\,\,\mathtt{then}\,\,\mathtt{#2}\,\,\mathtt{else}\,\,\mathtt{#3}}
\newcommand{\term}[1]{\mathtt{#1}}
\newcommand{\suc}[1]{\mathtt{succ}\,\,\mathtt{#1}}
\newcommand{\pred}[1]{\mathtt{pred}\,\,\mathtt{#1}}
\newcommand{\iszero}[1]{\mathtt{iszero}\,\,\mathtt{#1}}
\newcommand{\true}{\mathtt{true}}
\newcommand{\false}{\mathtt{false}}
\newcommand{\wrong}{\mathtt{wrong}}
\newcommand{\badnat}{\mathtt{badnat}}
\newcommand{\badbool}{\mathtt{badbool}}
\newcommand{\sstep}[2]{#1 \rightarrow #2}
\newcommand{\mstep}[2]{#1 \rightarrow^* #2}

\newcommand{\sarr}[2]{\rightarrow}
\newcommand{\marr}{\rightarrow^*}


\titlespacing{\subsubsection}{0pt}{\parskip}{-\parskip}
\titlespacing{\section}{0pt}{*4}{*1.5}

\begin{document}

\subsection*{Definition of Terms}
Define the set of terms as 
\begin{align*}
    S_0 &= \emptyset \\
    S_{i+1} &= \{\mathtt{true}, \mathtt{false}, 0\} 
    \cup \set{\suc{t_1}, \pred{t_1}, \iszero{t_1}}{\mathtt{t_1} \in S_i} 
    \cup \set{\ifthenelse{t_1}{t_2}{t_3}}{\mathtt{t_1}, \mathtt{t_2}, \mathtt{t_3} \in S_i} \\
    S &= \bigcup_i S_i
\end{align*}
\subsubsection*{Exercise 3.2.4}
How many elements does $S_3$ have?
\newline
\textbf{Solution:} $S_0 = \emptyset$, so $S_1 = \{\mathtt{true}, \mathtt{false}, 0\}$.  $S_2$ contains
the atomic terms from $S_1$ and also the compound terms built from atomic terms. The functions \texttt{suc},
\texttt{pred}, and \texttt{iszero} are all unary, so for each term $\mathtt{t}$ they produce a distinct
term $\tau\,\, \mathtt{t}$ where $\tau$ is one of \texttt{succ}, \texttt{pred}, and \texttt{iszero}.  Each function produces $3$ terms from the atomic terms, so in total $3 \cdot 3$ terms are produced by the unary functions.

The $\ifthenelse{t_1}{t_2}{t_3}$ function is ternary, and produces a distinct term for any choice of 
three terms $\mathtt{t_1}, \mathtt{t_2}, \mathtt{t_3}$. 
In the case of $S_2$ there are three choices (\texttt{true}, \texttt{false}, $0$) for each of
$\mathtt{t_1}, \mathtt{t_2}, \mathtt{t_3}$, so $\ifthenelse{t_1}{t_2}{t_3}$ produces $3 \cdot 3 \cdot 3 = 27$ distinct terms.  

In total, $S_2$ has $3 + 9 + 27 = 39$ terms.  Following the same approach, for $S_3$, the unary functions
produce $39 \cdot 3 = 117$ terms, and the ternary function produces $39^3 = 59,319$ terms.  Combining with
the three atomic terms gives $|S_3| = 3 + 117 + 59,319 = 59,439$.

\subsubsection*{Exercise 3.2.5}
Show that the sets $S_i$ are cumulative.
\begin{proof} We will show this by induction on $i$ that $S_i \subseteq S_{i+1}$.
\begin{description}
    \item[Base case:] $S_0 = \emptyset$ so trivially $S_0 \subseteq S_1$.
    \item[Inductive step:] Assume that we have shown that $S_{i-1} \subseteq S_i$.  We will show that
        $S_i \subseteq S_{i+1}$.  Let $\mathtt{t} \in S_i$ be some term.  Then, either $\mathtt{t}$ is and
        atomic term, or it is a compound term containing functions.  If $\mathtt{t}$ is atomic, then by 
        construction, $\mathtt{t} \in S_{i+1}$.  So, suppose $\mathtt{t}$ is a term containing functions.
        
        If the outermost function of $\mathtt{t}$ is a unary function then $\mathtt{t} = \tau\,\, \mathtt{s}$ where $\tau$ is a unary function and $\mathtt{s} \in S_{i-1}$.  By the IH, $\mathtt{s} \in S_i$ 
and it follows that $\mathtt{t} = \tau\,\, \mathtt{s} \in S_{i+1}$.

If the outermost function of $\mathtt{t}$ is the ternary conditional function, then a completely analogous
argument with $\mathtt{s}$ replaced by three terms $\mathtt{s_1}, \mathtt{s_2}, \mathtt{s_3}$ shows that
$\mathtt{t} \in S_{i+1}$.
\end{description}
By induction, it follows that $S_i \subseteq S_{i+1}$ for all $i \in \nn$ which shows that the sets
$S_i$ are cumulative.
\end{proof}

\subsubsection*{Exercise 3.3.4}
Suppose $P$ is a predicate on terms.  Show that Structural Induction holds:\newline
If, for each term $\term{s}$, given $P(\term{r})$ for all immediate subterms $\term{r}$ of $\term{s}$
we can show $P(\term{s})$, then $P(\term{s})$ holds for all $\term{s}$.

\begin{proof}
    First, note that the relation $\prec$ defined on terms by 
    \[ \term{r} \prec \term{s} \text{ iff } \term{r} \text{ is a subterm of } \term{s} \]
    is well-founded.  This can be seen by considering the concrete definition of terms using the sets
    $S_i$.  We showed in Exercise 3.2.5 that these sets are cumulative, and moreover by their definition,
    it is easy to see that they are all finite.  Since any term $\term{t}$ appears in some set $S_i$,
    it follows that any $\term{t}$ can only have finitely
    many $\prec$-predecessors.  Hence, there are no infinite $\prec$-descending chains.

    Now we can prove the desired statement by contradiction.  Suppose we have a predicate $P$ satisfying the
    hypothesis of Structural Induction.  We will show that $P(\term{s})$ holds for all $\term{s}$.  Suppose
    this was not the case, so for some term $\term{s}$, $P(\term{s})$ does not hold.  Using the 
    well-foundedness of $\prec$ choose such an $\term{s}$ which is $\prec$-minimal.  That is, $P(\term{s})$ 
    does not hold, but $P(\term{r})$ does hold for all terms $\term{r}$ such that $\term{r} \prec \term{s}$,
    i.e. all subterms of $\term{s}$.  In particular, $P$ holds for all immediate subterms of $\term{s}$.
    By our assumption that $P$ satisfies the hypothesis of structural induction, it follows that $P(\term{s})$ does hold. Contradiction. Hence, there are no terms $\term{s}$ such that $P(\term{s})$ does not hold.
\end{proof}


\subsubsection*{Exercise 3.5.5}
Spell out the induction principle used in the proof of Theorem 3.5.4.\newline
\textbf{Solution:} Suppose $P$ is a predicate evalaution statement derivations.  If $D$ is a derivation 
and assuming $P(\overline{D})$ holds for all subderivations $\overline{D}$ of $D$ we can prove $P(D)$,
then $P(D)$ holds for all derivations $D$.

\subsubsection*{Exercise 3.5.10}
Rephrase Definition 3.5.9 of the \emph{multi-step evaluation} relation $\rightarrow^*$ as a set of inference
rules.\newline
\textbf{Solution:} 
\begin{mathpar}
    \inferrule{\sstep{\term{t}}{\term{t'}}} {\mstep{\term{t}}{\term{t'}}}\quad \inferrule{}{\mstep{\term{t}}{\term{t}}} \quad \inferrule{\mstep{\term{t}}{\term{t'}} \text{ and } \mstep{\term{t'}}{\term{t''}}}{\mstep{\term{t}}{\term{t''}}}
\end{mathpar}


\subsubsection*{Exercise 3.5.13}
\begin{enumerate}
    \item[1.] Adding the rule
        \[ \sstep{\ifthenelse{\true}{\term{t_2}}{\term{t_3}}}{\term{t_3}} \]
        would make 
        \begin{itemize}
            \item Theorem 3.5.4 (Determinacy of one-step evalaution) fail since there are two posibilities
                for transitioning with $\ifthenelse{\true}{\cdot}{\cdot}$.
            \item Theorem 3.5.7 (every value is in normal form) will still hold
            \item Theorem 3.5.8 (normal forms are values) will still hold
            \item Theorem 3.5.11 (uniqueness of normal forms) will fail like Theorem 3.5.4
        \end{itemize}
    \item Adding the rule
        \begin{mathpar}
            \inferrule
            {
                \sstep{\term{t_2}}{\term{t_2'}}
            }
            {   \sstep
                {\ifthenelse{\term{t_1}}{\term{t_2}}{\term{t_3}}}
                {\ifthenelse{\term{t_1}}{\term{t_2'}}{\term{t_3}}}
            }
        \end{mathpar}
        would make 
        \begin{itemize}
            \item Theorem 3.5.4 (Determinacy of one-step evalaution) fail since there are two posibilities
                for transitioning with $\ifthenelse{\term{t1}}{\term{t2}}{\cdot}$.
            \item Theorem 3.5.7 (every value is in normal form) will still hold
            \item Theorem 3.5.8 (normal forms are values) will still hold
            \item Theorem 3.5.11 (uniqueness of normal forms) will still hold.  Suppose that
                $\mstep{\ifthenelse{\term{t_1}}{\term{t_2}}{\term{t_3}}}{\term{v}}$ where $\mstep{\term{t_2}}{\term{v}}$. 
                In the original semantics, this would only be possible by first evaluating 
                $\mstep{\term{t_1}}{\true}$ producing $\term{t_2}$ then evaluating 
                $\mstep{\term{t_2}}{\term{v}}$.  With the new rule, it is possible that some steps of 
                the evalaution $\mstep{\term{t_2}}{\term{v}}$ are interleaved with the evaluation 
                of $\mstep{\term{t_1}}{\true}$. Intuitively, this reorganization will not change the 
                resulting value, but a complete proof of this would require proving the diamond property.
                See the book for a full argument.
        \end{itemize}
\end{enumerate}

\subsubsection*{Exercise 3.5.14}
Proof Theorem 3.5.4 on arithmetic expressions: if $\sstep{\term{t}}{\term{t'}}$ and $\sstep{\term{t}}{\term{t''}}$ then $\term{t'} = \term{t''}$.
\begin{proof}
    By induction on the derivation of $\sstep{\term{t}}{\term{t'}}$.  If the last rule in the derivation
    is E-SUCC, then $\term{t}$ has the form $\suc{\term{t_1}}$ and $\sstep{\term{t_1}}{\term{t_1'}}$.
    It follows that the only rule
    that can apply to $\term{t}$ is E-SUCC, so the last rule of the derivation of 
    $\sstep{\term{t}}{\term{t''}}$ is E-SUCC as well and $\sstep{\term{t_1}}{\term{t_1''}}$.
    By the IH, $\term{t_1'} = \term{t_1''}$, and it follows that $\term{t'} = \term{t''}$.

    Now, suppose the last rule of the derivation is E-PRED.  So, $\term{t}$ has the form
    $\pred{\term{t_1}}$ and $\sstep{\term{t_1}}{\term{t_1'}}$.  Note that the rules E-PREDZERO
    and E-PREDSUCC cannot apply to $\term{t}$ because $\term{t_1}$ is not a value.  It follows
    that the last rule applied in the derivation of $\sstep{\term{t}}{\term{t''}}$ must be 
    E-PRED and $\sstep{\term{t_1}}{\term{t_1'}}$.  
    By the IH, $\term{t_1'} = \term{t_1''}$, and it follows that $\term{t'} = \term{t''}$.

    The argument if the last rule of the derivation is E-ISZERO follow in a completely analogous manner,
    except that we need to rule out the possibilites the possibilities of E-ISZEROZERO and
    E-ISZEROSUCC instead of E-PREDZERO and E-PREDSUCC.

    The arguments for the remaining rules are even simpler.  For example, suppose the final rule 
    used in the derivation is E-PREDZERO.  Then, $\term{t}$ is  $\pred{0}$ and trivially, 
    $\term{t'} = 0 = \term{t''}$. Similarly, if the final rule is E-ISZEROZERO we can show
    that $\term{t'} = \true = \term{t''}$.  Analogous reasoning works for E-PREDSUCC and E-ISZEROSUCC.
    In the case of E-PREDSUCC we show that $\term{t'} = \term{nv1} = \term{t''}$ and in the case of
    E-ISZEROSUCC we show that $\term{t'} = \false = \term{t''}$.
\end{proof}

\subsubsection*{Exercise 3.5.16}
Show that the two treatments of runtime errors agree.
\begin{claim}
    Let $\term{t}$ be a term in the original syntax and let $\sarr_1, \marr_1$ 
    denote the transition relations for the original
    semantics and $\sarr_2, \marr_2$ denote the transition relation for the augmented semantics.  
    Then, $\term{t} \marr_1 \term{t'}$ where $\term{t'}$ is stuck iff $\term{t} \marr_2 \wrong$.
\end{claim}

\begin{lem}\label{lem:samesteps}
    If $\term{t} \marr_1 \term{t'}$ then $\term{t} \marr_2 \term{t'}$ and follows the same steps.
\end{lem}
\begin{proof} If $\term{t'}$ can be transitioned to from $\term{t}$, then $\badnat$ or $\badbool$
    could never appear in a term that would allow one of the WRONG evaluations to apply.  Hence,
    the transition $\term{t} \marr_2 \term{t'}$ would follow exactly the same steps as 
    $\term{t} \marr_1 \term{t'}$.
\end{proof}

\begin{lem}
    If $\term{t}$ is stuck in the original semantics then $\term{t} \marr_2 \wrong$.
\end{lem}
\begin{proof} Suppose $\term{t}$ is stuck.  We show this by induction on the structure of $\term{t}$.
    There are a few cases to consider.
    \begin{description}
        \item[$\term{t}$ is of the form $\pred{\term{t_1}}$]
            If $\term{t_1}$ is not a value, then $\pred{\term{t_1}}$ is stuck iff $\term{t_1}$ is stuck.
            By the IH, $\term{t_1} \marr_2 \wrong$, so $\term{t} \marr_2 \wrong$ via E-PRED-WRONG.
            If $\term{t_1}$ is a value $\term{v}$, then
            since $\term{t}$ is stuck $\term{v}$ must not be a numeric value.  Hence, $\term{v}$ is 
            $\true$ or $\false$.  Thus, $\term{v}$ is a $\badnat$ and using E-PRED-WRONG
            $\term{t} \marr_2 \wrong$.
        \item[$\term{t}$ is of the form $\iszero{\term{t_1}}$]
            By the same reasoning as the previous case, 
            $\term{t} \marr_2 \wrong$ using E-ISZERO-WRONG.
        \item[$\term{t}$ is of the form $\suc{\term{t_1}}$]
            Similar to the previous cases, but we need not consider the case when $\term{t_1}$ is a value.
        \item[$\term{t}$ is of the form $\ifthenelse{\term{t_1}}{\term{t_2}}{\term{t_3}}$]
            If $\term{t_1}$ is not a value, then $\ifthenelse{\term{t_1}}{\term{t_2}}{\term{t_3}}$
            is stuck iff $\term{t_1}$ is stuck.  
            By the IH, $\term{t_1} \marr_2 \wrong$, so $\term{t} \marr_2 \wrong$ via E-IF-WRONG.
            If $\term{t_1}$ is a value $\term{v}$, then 
            since $\term{t}$ is stuck $\term{v}$ must be a numeric value.  Hence, $\term{v}$ is 
            a $\badbool$ and using E-IF-WRONG $\term{t} \marr_2 \wrong$.
    \end{description}
\end{proof}

The two previous lemmas show that if $\term{t} \marr_1 \term{t'}$ where $\term{t'}$ is stuck,
then $\term{t} \marr_2 \wrong$.  For the reverse direction, we show

\begin{lem} Suppose $\term{t} \marr_1 \term{t'}$ and $\term{t'} \sarr_2 \term{t''}$ where $\term{t''}$
    has $\wrong$ as a subterm, then $\term{t'}$ is stuck.
\end{lem}
\begin{proof} By induction on a derivation of $\term{t'} \sarr_2 \term{t''}$.
    Suppose that the final rule in the derivation is E-IF-WRONG. Then, $\term{t'}$ has the form
    $\ifthenelse{\badbool}{\term{t_2}}{\term{t_3}}$.  Because $\term{t'}$ was obtained through evaluation
    in the original semantics, it follows that $\badbool$ is a numeric value $\term{nv}$ and not $\wrong$.
    Hence, $\term{t'}$ is stuck.
    We can argue similarly if the final rule is E-SUCC-WRONG, E-PRED-WRONG, or E-ISZERO-WRONG.  

    If the final rule in the derivation is E-PRED, then $\term{t'}$ is of the form
    $\pred{\term{t_1}}$ and $\term{t_1} \sarr_2 \term{t_1'}$.  By our assumption that $\term{t''}$
    has $\wrong$ as a subterm it follows that the evaluation $\term{t_1} \sarr_2 \term{t_1'}$ must use
    one of the $\wrong$ producing rules E-IF-WRONG, E-SUCC-WRONG, E-PRED-WRONG, E-ISZERO-WRONG.  From the 
    induction hypothesis, we get that $\term{t_1}$ is stuck which implies that 
    $\term{t'} = \pred{\term{t_1}}$ is stuck.  We can argue in a similar manner if the final rule in
    the derivation is E-SUCC, E-ISZERO, or E-IF.

    The remaining numeric expression produce values, so could not be the final rule in the derivation.
    The rules E-IFTRUE and E-IFFALSE extract a subterm from $\term{t'}$, so they could not have $\wrong$
    as a subterm since the derivation $\term{t} \marr_1 \term{t'}$ happens in the original semantics.
    Hence, they could not be the final rule of the derivation either.
\end{proof}

Combining the previous lemma and Lemma \ref{lem:samesteps}, we can prove the reverse direction.
Suppose that $\term{t} \marr_2 \wrong$.  By Lemma \ref{lem:samesteps} an initial segment of the 
multi-step evaluation is done purely in the original semantics.  Let $\term{t'}$ be the final term produced
using the original evaluation rules.  It follows that the next term in the evaluation of 
$\term{t} \marr_2 \wrong$ must have $\wrong$ as a subterm, so using the previous lemma, 
$\term{t'}$ is stuck.


\subsubsection*{Exercise 3.5.17}
Show that the small-step and big-step semantics agree, i.e. $\term{t} \marr \term{v}$ iff 
$\term{t} \Downarrow \term{v}$.

\begin{proof} We show this by induction on the structure of $\term{t}$.
    \begin{description}
    \item[$\term{t}$ is a value] Trivially, $\term{t} \marr \term{v}$ iff $\term{t} \Downarrow \term{v}$.
    \item[$\term{t}$ is $\ifthenelse{\term{t_1}}{\term{t_2}}{\term{t_3}}$:] By the induction hypothesis,
        $\term{t_i} \marr \term{v}$ iff $\term{t_i} \Downarrow \term{v}$ for $i = 1, 2, 3$.
        For the forward direction, assume that $\term{t} \marr \term{v}$.  Necessarily,
        $\term{t_i} \marr \term{v_i}$ for $i = 1, 2, 3$.  Consider the following evaluation of
        $\term{t}$.  First, use the evaluation $\term{t_1} \marr \term{v_1}$ and E-IF to obtain 
        $\term{t} \marr \ifthenelse{\term{v_1}}{\term{t_2}}{\term{t_3}}$.  If $\term{v_1} = \true$ we 
        next apply E-IFTRUE to obtain $\term{t_2}$ and proceed with $\term{t_2} \marr \term{v_2}$ to obtain
        $\term{t} \marr \term{v_2}$.  Otherwise, $\term{v_1} = \false$ and we use E-IFFALSE and 
        $\term{t_3} \marr \term{v_3}$ to obtain $\term{t} \marr {\term{v_3}}$.
        By the uniqueness of normal forms (Theorem 3.5.11), this proposed evaluation is valid and
        results in the correct value. Suppose the former case $\term{t_1} \marr \true$ holds.  Then, 
        by the IH, $\term{t_1} \Downarrow \true$ and $\term{t_2} \Downarrow \term{v_2}$, so
        B-IFTRUE gives $\term{t} \Downarrow \term{v_2}$.  In the latter case $\term{t_1} \marr \false$,
        and B-IFFALSE gives $\term{t} \Downarrow \term{v_3}$.

        For the reverse direction, assume that $\term{t} \Downarrow \term{v}$.  So, either B-IFTRUE or
        B-IFFALSE were used to evaluate $\term{t}$.  If B-IFTRUE was used, then 
        $\term{t_1} \Downarrow \true$, $\term{t_2} \Downarrow \term{v_2}$, and 
        $\term{t} \Downarrow \term{v_2}$ for some $\term{v_2}$.  
        By the induction hypothesis, $\term{t_1} \marr \true$ and
        $\term{t_2} \marr \term{v_2}$.  We can again appeal to the uniqueness of normal forms theorem
        to construct a valid evaluation of $\term{t} \marr \term{v_2}$ following the same approach as 
        above. If B-IFFALSE was used, then
        $\term{t_3} \Downarrow \term{v_3}$ for some $\term{v_3}$
        so that $\term{t} \Downarrow \term{v_3}$, and we again
        construct a valid evaluation producing giving $\term{t} \marr \term{v_3}$.
    \item[$\term{t}$ is $\pred{t_1}$:] By the induction hypothesis,
        $\term{t_i} \marr \term{v}$ iff $\term{t_i} \Downarrow \term{v}$ for $i = 1, 2, 3$.
        For the forward direction, assume that $\term{t} \marr \term{v}$.  Hence, 
        $\term{t_1} \marr \term{v_1}$ for some value $\term{v_1}$.  Following the approach for
        if-then-else term, we will construct a valid evaluation of $\term{t}$.  If $\term{v_1} = 0$,
        then we can evaluate $\term{t}$ by first using $\term{t_1} \marr 0$ and E-PRED to obtain
        $\term{t} \marr \pred{0}$.  Then, we can apply E-PREDZERO to obtain $\term{t} \marr 0$.
        On the other hand, if $\term{v_1} = \suc{\term{nv}}$, then we would use E-PREDSUCC in the final
        step to obtain $\term{t} \marr \term{nv}$.  In the former case, the induction hypothesis gives
        $\term{t_1} \Downarrow 0$ so applying B-PREDZERO to $\term{t}$ gives $\term{t} \Downarrow 0$.
        In the later case, the induction hypothesis gives $\term{t_1} \Downarrow \suc{\term{nv}}$ and
        applying B-PREDSUCC gives $\term{t} \Downarrow \term{nv}$ as required.

        For the reverse direction, assume that $\term{t} \Downarrow \term{v}$.  
        If B-PREDZERO is applied then $\term{t_1} \Downarrow 0$ so by the IH, $\term{t_1} \marr 0$.
        Hence, we can obtain $\term{t} \marr 0$ using E-PRED with $\term{t_1} \marr 0$ and applying
        E-PREDZERO in the final step.  If B-PREDSUCC is applied then $\term{t_1} \Downarrow \term{nv}$
        and $\term{t_1} \marr \term{nv}$ by the IH.  As in the B-PREDZERO case, we can use this
        to obtain $\term{t} \marr \term{nv}$.

    \item[$\term{t}$ is $\suc{\term{t_1}}$ or $\iszero{\term{t_1}}$:]
        Similar to the $\term{pred}$ case.
\end{description}
\end{proof}

\subsubsection*{Exercise 3.5.18}
Give evaluation rules so that the $\term{then}$ and $\term{else}$ branches of an $\term{if}$ expression
are evaluated before the guard is evaluated.\newline
\textbf{Solution:} Remove E-IFTRUE, E-IFFALSE, and E-IF and replace with
\begin{mathpar}
    \inferrule{}{
    \sstep{\ifthenelse{\true}{\term{v_2}}{\term{v_3}}}{\term{v_2}}
    }

    \inferrule{}{
    \sstep{\ifthenelse{\false}{\term{v_2}}{\term{v_3}}}{\term{v_3}}}

    \inferrule{
        \sstep{\term{t_2}}{\term{t_2'}}
    }{
    \sstep{\ifthenelse{\term{t_1}}{\term{t_2}}{\term{t_3}}}
        {\ifthenelse{\term{t_1}}{\term{t_2'}}{\term{t_3}}}
    }

    \inferrule{
        \sstep{\term{t_3}}{\term{t_3'}}
    }{
    \sstep{\ifthenelse{\term{t_1}}{\term{v_2}}{\term{t_3}}}
        {\ifthenelse{\term{t_1}}{\term{v_2}}{\term{t_3'}}}
    }

    \inferrule{
        \sstep{\term{t_1}}{\term{t_1'}}
    }{
    \sstep{\ifthenelse{\term{t_1}}{\term{v_2}}{\term{v_3}}}
            {\ifthenelse{\term{t_1'}}{\term{v_2}}{\term{v_3}}} 
    }
\end{mathpar}

\end{document}
