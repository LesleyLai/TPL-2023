\documentclass[12pt]{article}
\setlength{\oddsidemargin}{0in}
\setlength{\evensidemargin}{0in}
\setlength{\textwidth}{6.5in}
\setlength{\parindent}{0in}
\setlength{\parskip}{\baselineskip}

\usepackage{amsmath,amsthm,amsfonts,amssymb,fancyvrb}
\usepackage{mathpartir}

% Helpful macros
\newcommand{\mt}[1]{\ensuremath{\text{#1}}}
\newcommand{\isA}[2]{\ensuremath{#1 \; #2}}
\newcommand{\isANat}[1]{\isA{#1}{\mt{nat}}}
\newcommand{\zero}{\mt{zero}}
\newcommand{\mySucc}[1]{\mt{succ}(#1)}


%% Judgments


\begin{document}

TPL \hfill Chapter 3 Exercise\\
Your Name

\hrulefill

\subsection*{Questions}

\begin{enumerate}

\item[3.2.4]
\item[3.2.5]
\item[3.5.10]
\item[3.5.13]
\item[3.5.16]
\item[3.5.17]
\item[3.5.18]


\end{enumerate}

\newpage

\subsection*{Appendix}
Here is an example latex typesetting of an inductive definition. You can copy/paste and modify this code. After finishing all the exercises, feel free to delete this section.

\begin{mathpar}
\inferrule{
}{
    \isANat{\zero}
}

\inferrule{
    \isANat{n}
}{
    \isANat{\mySucc{n}}
}
\end{mathpar}

You can nest \textbackslash inferrule* for derivation tree.
For example:

\begin{mathpar}
\inferrule*{
\inferrule*{
  \text {empty tree} \text {empty tree}
}{
  \text {node}(\text {empty;empty}) \text{ tree}
}
\quad
\text {empty tree}
}
{
\text {node}( \text {node}(\text {empty;empty}); \text{empty}) \text{ tree}
}
\end{mathpar} 


\end{document}
