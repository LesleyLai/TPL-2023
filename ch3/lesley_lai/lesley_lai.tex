\documentclass[12pt]{article}
\setlength{\oddsidemargin}{0in}
\setlength{\evensidemargin}{0in}
\setlength{\textwidth}{6.5in}
\setlength{\parindent}{0in}
\setlength{\parskip}{\baselineskip}

\usepackage{amsmath,amsthm,amsfonts,amssymb,fancyvrb}
\usepackage{mathpartir}

% Helpful macros
\newcommand{\mt}[1]{\ensuremath{\text{#1}}}
\newcommand{\isA}[2]{\ensuremath{#1 \; #2}}
\newcommand{\isANat}[1]{\isA{#1}{\mt{nat}}}
\newcommand{\zero}{\mt{zero}}
\newcommand{\mySucc}[1]{\mt{succ}(#1)}


%% Judgments


\begin{document}

TPL \hfill Chapter 3 Exercise\\
Lesley Lai

\hrulefill

\subsection*{Questions}

\begin{enumerate}

\item[3.2.4] \textit{How many elements does $S_3$ have?}

\begin{align*}
|S_{i+1}| &= 3 + 3 \times |S_i| + |S_i|^3 \\
|S_0| &= 0 \\
|S_1| &= 3 \\
|S_2| &= 3 + 9 + 27 = 39 \\
|S_3| &= 3 + 39 \times 3 + 39^3 = 59439
\end{align*}

\item[3.5.10] \textit{Rephrase Definition 3.5.9 as a set of inference rules.}

\begin{mathpar}
\inferrule{
    t \longrightarrow t'
}{
    t \longrightarrow^* t'
}

\inferrule{
}{
    t \longrightarrow^* t
}

\inferrule{
    t \longrightarrow^* t' \quad
    t' \longrightarrow^* t''
}{
    t \longrightarrow^* t''
}
\end{mathpar}

\newpage
\item[3.5.13]
\begin{enumerate}
\item[1.] \textit{Suppose we add a new rule}
$$
\text{if true then } t_2 \text{ else } t_3 \longrightarrow t_3 \qquad (\text {E-Funny1})
$$

\textit{to the ones in Figure 3-1. Which of the above theorems (3.5.4, 3.5.7, 3.5.8, 3.5.11, and 3.5.12) remain valid?}

\begin{itemize}
\item 3.5.4 (Determinacy of one-step evaluation) is \textbf{invalid} since there are two rules applied to $\text{if true then } t_2 \text{ else } t_3$
\item 3.5.7 (every value is normal form) is still \textbf{valid}
\item 3.5.8 (If t is in normal form, then t is a value) is still \textbf{valid}
\item 3.5.11 (Uniqueness of normal forms) is \textbf{invalid} because evaluation is no longer deterministic
\item 3.5.12 (Termination of Evaluation) is still \textbf{valid}
\end{itemize}

\item[2.] \textit{Suppose instead that we add this rule:}

\begin{mathpar}
\inferrule{
    t_2 \longrightarrow t_2'
}{
\text{if } t_1 \text{ then } t_2 \text{ else } t_3 \longrightarrow \text{if } t_1 \text{ then } t_2' \text{ else } t_3
}
\quad (\text{E-Funny2})
\end{mathpar}

\textit{Now which of the above theorems remain valid? Do any of the proofs need to change?}

\begin{itemize}
\item 3.5.4 (Determinacy of one-step evaluation) is \textbf{invalid}
\item 3.5.7 (every value is normal form) is still \textbf{valid}
\item 3.5.8 (If t is in normal form, then t is a value) is still \textbf{valid}
\item 3.5.11 (Uniqueness of normal forms) is \textbf{valid}. The proof need to change since the single-step evaluation is no longer deterministic.
\item 3.5.12 (Termination of Evaluation) is still \textbf{valid}
\end{itemize}

\end{enumerate}

\newpage

\item[3.5.16]
TODO

\item[3.5.17] \textit{Show that the small-step and big-step semantics for this language coincide, i.e. $t \longrightarrow^* v \text{ iff } t \Downarrow v$} \\

We need to show both side of an iff relationship.

\begin{itemize}
\item Prove $t \longrightarrow^* v \text{ if } t \Downarrow v$

TODO

\item Prove $t \Downarrow v  \text{ if } t \longrightarrow^* v$

TODO
\end{itemize}




\item[3.5.18] \textit{Suppose we want to change the evaluation strategy of our
language so that the \texttt{then} and \texttt{else} branches of an \texttt{if} expression are evaluated (in that order) before the guard is evaluated. Show how the evaluation rules need to change to achieve this effect.}

We need to replace all the boolean evaluation rules with the following rules:
\begin{mathpar}
\inferrule{
    t_2 \longrightarrow t_2'
}{
\text{if } t_1 \text{ then } t_2 \text{ else } t_3 \longrightarrow \text{if } t_1 \text{ then } t_2' \text{ else } t_3
}
\quad (\text{E-Then})

\inferrule{
    v_2 \text{ val} \quad t_3 \longrightarrow t_3'
}{
\text{if } t_1 \text{ then } v_2 \text{ else } t_3 \longrightarrow \text{if } t_1 \text{ then } v_2 \text{ else } t_3'
}
\quad (\text{E-Else})

\inferrule{
    v_2 \text{ val}\quad v_3 \text{ val} \quad t_1 \longrightarrow t_1'
}{
\text{if } t_1 \text{ then } v_2 \text{ else } v_3 \longrightarrow \text{if } t_1' \text{ then } v_2 \text{ else } v_3
}
\quad (\text{E-If})

\inferrule{
    v_2 \text{ val} \quad v_3 \text{ val}
}{
\text{if true then } v_2 \text{ else } v_3 \longrightarrow v_2
}
\quad (\text{E-IfTrue})

\inferrule{
    v_2 \text{ val} \quad v_3 \text{ val}
}{
\text{if false then } v_2 \text{ else } v_3 \longrightarrow v_3
}
\quad (\text{E-IfFalse})
\end{mathpar}

\end{enumerate}

\newpage

\end{document}
