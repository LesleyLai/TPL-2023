\documentclass[12pt]{article}
\setlength{\oddsidemargin}{0in}
\setlength{\evensidemargin}{0in}
\setlength{\textwidth}{6.5in}
\setlength{\parindent}{0in}
\setlength{\parskip}{\baselineskip}

\usepackage{amsmath,amsthm,amsfonts,amssymb,fancyvrb}
\usepackage{mathpartir}

% Helpful macros
\newcommand{\mt}[1]{\ensuremath{\text{#1}}}
\newcommand{\isA}[2]{\ensuremath{#1 \; #2}}
\newcommand{\isANat}[1]{\isA{#1}{\mt{nat}}}
\newcommand{\zero}{\mt{zero}}
\newcommand{\mySucc}[1]{\mt{succ}(#1)}


%% Judgments


\begin{document}

TPL \hfill Chapter 5 Exercise\\
Your Name

\hrulefill

\subsection*{Questions}

\begin{enumerate}

\item[5.2.1] Define logical \texttt{or} and \texttt{not} functions

\item[5.2.4] Define a term for raising one number to the
power of another.

\item[5.2.7] Write a function \texttt{equal} that tests two numbers for equality and returns a Church boolean. For example,

\begin{align*}
& \text { equal } \ c_3 \ c_3 ; \\
>\ & (\lambda t . \ \lambda f . \ t) \\
& \text { equal } \ c_3 \ c_2 ; \\
>\ & (\lambda t . \ \lambda f . \ f)
\end{align*}

\item[5.2.8]

A list can be represented in the lambda calculus by its \texttt{fold} function. (OCaml's name for this function is \texttt{fold\_left}; it is also sometimes called \texttt{reduce}.) For example, the list \texttt{[x,y,z]} becomes a function that takes two arguments \texttt{c} and \texttt{n} and returns \texttt{c x (c y (c z n)))}. What would the representation of \texttt{nil} be? Write a function \texttt{cons} that takes an element \texttt{h} and a list (that is, a \texttt{fold} function) \texttt{t} and returns a similar representation of the list formed by prepending \texttt{h} to \texttt{t}. Write \texttt{isnil} and \texttt{head} functions, each taking a list parameter. Finally, write a \texttt{tail} function for this representation of lists (this is quite a bit harder and requires a trick analogous to the one used to define \texttt{prd} for numbers).

\item[5.2.11] Use \texttt{fix} and the encoding of lists from Exercise 5.2.8 to write a function that sums lists of Church numerals

\item[5.3.6] Adapt these rules to describe the other three strategies for
evaluation—full beta-reduction, normal-order, and lazy evaluation.

\item[5.3.8] Exercise 4.2.2 introduced a ``big-step" style of evaluation for
arithmetic expressions, where the basic evaluation relation is ``term \texttt{t} evaluates to final result \texttt{v}." Show how to formulate the evaluation rules for lambda-terms in the big-step style.

\end{enumerate}

\newpage

\subsection*{Appendix}
Here is an example latex typesetting of an inductive definition. You can copy/paste and modify this code. After finishing all the exercises, feel free to delete this section.

\begin{mathpar}
\inferrule{
}{
    \isANat{\zero}
}

\inferrule{
    \isANat{n}
}{
    \isANat{\mySucc{n}}
}
\end{mathpar}

You can nest \textbackslash inferrule* for derivation tree.
For example:

\begin{mathpar}
\inferrule*{
\inferrule*{
  \text {empty tree} \text {empty tree}
}{
  \text {node}(\text {empty;empty}) \text{ tree}
}
\quad
\text {empty tree}
}
{
\text {node}( \text {node}(\text {empty;empty}); \text{empty}) \text{ tree}
}
\end{mathpar} 


\end{document}
