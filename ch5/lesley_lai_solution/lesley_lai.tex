\documentclass[12pt]{article}
\setlength{\oddsidemargin}{0in}
\setlength{\evensidemargin}{0in}
\setlength{\textwidth}{6.5in}
\setlength{\parindent}{0in}
\setlength{\parskip}{\baselineskip}

\usepackage{amsmath,amsthm,amsfonts,amssymb,fancyvrb}
\usepackage{mathpartir}

% Helpful macros
\newcommand{\mt}[1]{\ensuremath{\text{#1}}}
\newcommand{\tru}{\mt{tru}}
\newcommand{\fls}{\mt{fls}}
\newcommand{\abstr}[1]{\lambda #1.\ }
\newcommand{\nil}{\mt{\texttt{nil}}}
\newcommand{\cons}{\mt{\texttt{cons}}}
\newcommand{\isnil}{\mt{\texttt{isnil}}}
\newcommand{\head}{\mt{\texttt{head}}}
\newcommand{\tail}{\mt{\texttt{tail}}}
\newcommand{\fst}{\mt{\texttt{fst}}}
\newcommand{\snd}{\mt{\texttt{snd}}}
\newcommand{\pair}{\mt{\texttt{pair}}}
\newcommand{\plus}{\mt{\texttt{plus}}}
\newcommand{\fix}{\mt{\texttt{fix}}}
\newcommand{\realbool}{\mt{\texttt{realbool}}}

\begin{document}

TPL \hfill Chapter 5 Exercise\\
Lesley Lai

\hrulefill

\subsection*{Questions}

\begin{enumerate}

\item[5.2.1] Define logical \texttt{or} and \texttt{not} functions

\begin{align*}
\mt{or} &= \abstr{b} \abstr{c} b \ \tru \ c; \\
\mt{not} &= \abstr{b} \abstr{t} \abstr{f} b \ f \ t ;
\end{align*}

Tests:
\begin{align*}
\mt{or} \ \tru \ \tru &= \tru \ \tru \ \tru = \tru \\
\mt{or} \ \fls \ \tru &= \fls \ \tru \ \tru = \tru \\
\mt{or} \ \tru \ \fls &= \tru \ \tru \ \fls = \tru \\
\mt{or} \ \fls \ \fls &= \fls \ \tru \ \fls = \fls \\
\mt{not} \ \tru &= \abstr{t}\abstr{f} \tru \ f \ t \\
&= \abstr{t} \abstr{f} f \\ &= \fls \\
\mt{not} \ \fls &= \abstr{t}\abstr{f} \fls \ f \ t \\
&= \abstr{t} \abstr{f} t \\ &= \tru
\end{align*}

\newpage

\item[5.2.4] Define a term for raising one number to the
power of another.

$$
\mt{power} = \abstr{m}\abstr{n} m\ (\mt{times}\ n)\ c_0;
$$

\item[5.2.7] Write a function \texttt{equal} that tests two numbers for equality and returns a Church boolean. For example,

\begin{align*}
& \mt{ equal } \ c_3 \ c_3 ; \\
>\ & (\abstr{t} \abstr{f} t) \\
& \mt { equal } \ c_3 \ c_2 ; \\
>\ & (\abstr{t} \abstr{f} f)
\end{align*}

Answer:
$$
\mt{equal} = \abstr{c_1}\abstr{c_2} \mt{and} \ (\mt{iszro} (c_1 \ \mt{prd} \ c_2)) \ (\mt{iszro} (c_2 \ \mt{prd} \ c_1));
$$

\newpage
\item[5.2.8]

A list can be represented in the lambda calculus by its \texttt{fold} function. (OCaml's name for this function is \texttt{fold\_left}; it is also sometimes called \texttt{reduce}.) For example, the list \texttt{[x,y,z]} becomes a function that takes two arguments \texttt{c} and \texttt{n} and returns \texttt{c x (c y (c z n)))}. What would the representation of \texttt{nil} be? Write a function \texttt{cons} that takes an element \texttt{h} and a list (that is, a \texttt{fold} function) \texttt{t} and returns a similar representation of the list formed by prepending \texttt{h} to \texttt{t}. Write \texttt{isnil} and \texttt{head} functions, each taking a list parameter. Finally, write a \texttt{tail} function for this representation of lists (this is quite a bit harder and requires a trick analogous to the one used to define \texttt{prd} for numbers).

\begin{align*}
\nil &= \abstr{c} \abstr{n} n; \\
\cons &= \abstr{h} \abstr{t} \abstr{c} \abstr{n} (c \ h \ (t \ c \ n));
\end{align*}

\begin{align*}
\isnil &= \abstr{l} l \ (\abstr{h} \abstr{t} \fls) \ \tru \\
\head &= \abstr{l} l  \ (\abstr{h} \abstr{t} h) \ \nil; \\
\tail &= \abstr{l} \fst \ (l \\
&\qquad (\abstr{h} \abstr{t} (\pair \ (\snd \ t) \ (\cons \ h \ (\snd \ t))) \\
&\qquad (\pair \ \nil \ \nil) \\
&\quad);
\end{align*}


\item[5.2.11] Use \texttt{fix} and the encoding of lists from Exercise 5.2.8 to write a function that sums lists of Church numerals

\begin{align*}
\mt{sum\_impl} &= \abstr{\mt{fct}} \abstr{l} \\ 
&\qquad \texttt{if} \ \realbool \ (\isnil \ l) \ \texttt{then} \\
&\qquad\quad c_0 \\
&\qquad \texttt{else} \\
&\qquad\quad \plus \ \mt{fct} \ (\tail \ l) \\
\mt{sum} &= \fix \ \mt{sum\_impl}
\end{align*}

\newpage

\item[5.3.8] Exercise 4.2.2 introduced a ``big-step" style of evaluation for
arithmetic expressions, where the basic evaluation relation is ``term \texttt{t} evaluates to final result \texttt{v}." Show how to formulate the evaluation rules for lambda-terms in the big-step style.

$$
\inferrule{
  \mt{t}_1 \Downarrow (\abstr{\mt{x}} \mt{t}_{12}) \quad \mt{t}_2 \Downarrow \mt{v}_2
}{ t_1 \ t_2 \Downarrow \mt{t}_{12}[\mt{v}_2/\mt{x}]
}
$$

\end{enumerate}

\newpage


\end{document}
